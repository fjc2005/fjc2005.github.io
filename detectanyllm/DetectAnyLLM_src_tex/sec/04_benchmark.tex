\section{Proposed MIRAGE Benchmark}
% Despite significant advancement in MGTD research in recent years, existing benchmarks still exhibit notable limitations in terms of Domain-Diversity and Source-LLMs Coverage.
% %
% The narrow range of text domains fails to reflect the diverse contexts in which LLMs are applied~\cite{turingbench, detectrl}, while the heavy reliance on small-scale, open-source models~\cite{m4bench, raid} neglects the widespread adoption of commercial, closed-source LLMs in practical applications.
% %
% Moreover, most existing benchmark datasets focus primarily on sampling for MGT while lacking consideration for MRT~\cite{mage, hc3}, which diverges from real-world scenarios.
% %
% These limitations hinder a thorough evaluation of current detection methods, resulting in models that perform well on specific datasets but struggle in broader, real-world applications.

% Current benchmarks often focus on narrow text domains~\cite{turingbench, detectrl} and primarily rely on small-scale, open-source models~\cite{m4bench, raid}, overlooking the prevalent use of commercial, closed-source LLMs in real-world applications. Furthermore, existing datasets predominantly emphasize MGT sampling while neglecting MRT~\cite{mage, hc3}, which misaligns with practical scenarios. 

Current benchmarks exhibit notable limitations in diversity of text domains~\cite{turingbench, detectrl}, coverage of source LLMs~\cite{m4bench, raid} and evaluation tasks~\cite{mage, hc3}.
%
These constraints hinder comprehensive assessment and lead to detection models with poor generalizability beyond narrowly defined benchmark scenarios.

In order to facilitate a robust evaluation that better reflect real-world application, we present the \textit{\textbf{M}ulti-domain \textbf{I}nclusive \textbf{R}ealistic \textbf{A}ssessment for machine \textbf{G}enerated text d\textbf{E}tection (\textbf{MIRAGE})} benchmark.
%
MIRAGE constitutes the most comprehensive multi-task MGTD evaluation framework to date, incorporating both generative and revisionary text across diverse domains, employing most advanced LLMs, including 13 proprietary and 4 open-source LLMs.


\subsection{Benchmark Construction}
 
\noindent \textbf{Multi-domain Sampling. }
Considering that LLMs exhibit varying performance across different text domains, MIRAGE sample HWT of 5 domains from 10 corpora, detail information are presented in Supplementary Material.

\noindent \textbf{Inclusive MGT Tasks. }
Following established methodologies in the ~\cite{fastdetectgpt} and ~\cite{imbd}, we designed three distinct MGT tasks: \textit{Generate}, \textit{Polish}, and \textit{Rewrite}.
%
The \textit{Generate} task involves creating new text based on the first 30 tokens of an HWT.
%
The \textit{Polish} task refines an existing HWT while preserving its original details and meaning.
%
The \textit{Rewrite} task paraphrases a given HWT without altering its meaning or essential details.
%
The detailed prompt of each task will be present in Supplementary Material.

\noindent \textbf{Realistic LLM Usage. }
In real-world scenarios, people typically rely on powerful commercial closed-source LLMs to generate or revise text.
%
However, most of existing benchmarks~\cite{turingbench, hc3, m4bench, mage, raid, detectrl} rely on open-source LLMs to create data, resulting in a gap between current benchmarks and real-world evaluations.
%
To address this, MIRAGE incorporates 13 mainstream commercial LLMs, as detailed in Supplementary Material.

Concurrently, recognizing the increasing deployment of high-performance open-source models in localized applications, we incorporated four advanced open-source LLMs~\cite{qwen2.5, llama3}, ensuring comprehensive coverage of the contemporary LLM ecosystem.

\noindent \textbf{Data Cleaning. }
We remove the `\textbackslash n' character from each data entry to prevent the detector from identifying MGT based on the presence of the `\textbackslash n' symbol. 
%
Subsequently, we segment the text into words using spaces and filter out texts containing 100-200 words from these datasets to control for length-based detection biases.

\noindent \textbf{Composition. }
We consider two distinct evaluation scenarios to reflect real-world applications:

\textit{\textbf{Disjoint-Input Generation(DIG):}} This assesses a detector's ability to distinguish between an original HWT and a corresponding MGT or MRT produced by a single LLM.

\textit{\textbf{Shared-Input Generation (SIG):}} This assesses a detector's capacity to simultaneously differentiate multiple MGTs or MRTs (produced by different LLMs) from the same original HWT.

%
We design each LLM to generate 2,000 samples for each MGT task, equally distributed between DIG and SIG scenarios (1,000 samples each).
%
To ensure the balance of text fields, both DIG and SIG maintained identical domain distribution ratios, as detailed in Supplementary Material.

Sampling begins with constructing domain level HWT datasets by proportionally merging source datasets within each domain.
%
Such dataset-mixing strategy ease dataset-bias by preventing oversampling from single dataset.

During implementation, SIG is treated as an independent ``model'' and incorporates alongside the 17 individual LLMs in the sampling process.
%
For each model (including SIG), we sequentially sample data from each domain dataset.
%
Once sampled, items are removed from corresponding domain datasets to maintain the distinction between DIG and SIG data.
%
Within each text domain, data is sampled continuously until the number of samples for that domain meets the requirements specified in Supplementary Material.
%
Once the data sampling for one text domain is complete, the process moves to the next, repeating until all text domains have been sampled.

This methodology produces the DIG dataset for each LLM and a comprehensive SIG dataset, which are subsequently combined to form the complete sample set for each LLM across all tasks.

\noindent \textbf{Data Augmentation. }
The language style is a key distinguishing feature between HWT and MGT, with a closer alignment to human language style posing a more challenging task for MGT detectors.
%
With this consideration, we introduce data augmentation in terms of LLM's language styles. 
%
Specifically, we incorporated the phrase ``in a <style> style" into the input prompt.
%
We manually select 16 different language styles, randomly choosing one during each LLM inference to achieve style-diversity.

This approach helps assess the robustness of detectors against content with varying language styles.
%
Moreover, in practice, one often adds language style constraints to prompts when using LLMs to assist with writing.
%
Thus, It further enhances the benchmark’s ability to simulate real-world scenarios.

\noindent \textbf{Post Cleaning. }
After generating MGT or MRT from the above HWT data, we perform data cleaning on the generated data.
%
First, all instances of the `\textbackslash n' and `\textbackslash r' are removed, to prevent detection from symbol's feature.
%
Next, we remove texts with fewer than 90 words or more than 220 words, to prevent the impact of text length variations on detection, and finally obtain the MIRAGE benchmark dataset.
%
The statistical results are presented in Supplementary Material.

\subsection{Evaluate Metrics}
Consistent with prior works~\cite{detectgpt, fastdetectgpt, imbd}, we adopt the \textit{Area Under the Receiver Operating Characteristic Curve (AUROC)} as the primary evaluation metric.
%
To assess the performance on specific threshold, we incorporate \textit{TPR at a 5\% false positive rate (TPR@5\%)} as a supplementary metric.
%
Furthermore, considering the MIRAGE-SIG is a class-imbalanced dataset, we additionally report the \textit{Matthews Correlation Coefficient (MCC)} and \textit{Balanced Accuracy} to provide a more comprehensive evaluation.
%
Together, this diverse set of metrics provides a comprehensive assessment of detector's performance, ensuring that the evaluation reflects both theoretical completeness and real-world applicability.
%
For more information on metrics, please see Supplementary Material.